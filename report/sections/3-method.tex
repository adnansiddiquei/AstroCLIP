%! Author = adnansiddiquei
%! Date = 04/06/2024

\section{Method}\label{sec:method}
Assuming that galaxy spectra and galaxy images can be viewed as different physical representations of the same underlying
object, reasonably, one can assume that there is enough mutual information within the two modalities to be able to effectively
embed them into a shared latent space.
As such, the objective of this piece of work was to reproduce the results of the original AstroCLIP paper~\citep{astroclip}
which implemented a multi-modal contrastive learning approach to embed galaxy spectra and galaxy images into a shared embedding
space.
This was done using 2 pretrained convolutional spectrum and image embedders, and then fine-tuning the model using contrastive
learning under the InfoNCE loss~\citep{oord2019}, as per Equation~\eqref{eq:infonce}, to learn the low-dimensional
embedding space for the two modalities

\begin{equation}
\label{eq:infonce}
    \mathcal{L}(\mathbf{X}, \mathbf{Y}) = - \frac{1}{K} \sum_{i=1}^K \log \frac{\exp(S_C(\mathbf{x}_i, \mathbf{y}_i) / \tau)}{\sum_{j}^K \exp(S_C(\mathbf{x}_i, \mathbf{y}_j) / \tau)}.
\end{equation}
where $\mathbf{X}$ and $\mathbf{Y}$ is an embedded batch of sets of galaxy spectra and images, $K$ is the size of the
batch, $S_C$ is the cosine similarity function, and $\tau$ is a smoothing parameter (often called the temperature).
Indices $i$ correspond to positive pairs (embedded image-spectra pairs which correspond to the same galaxy) and $j$ corresponds
to negative pairs (embedded image-spectra pairs which correspond to different galaxies).
Formerly, $S_C$ defined in Equation~\eqref{eq:cosine-similarity}

\begin{equation}
\label{eq:cosine-similarity}
    S_C(\mathbf{x}_i, \mathbf{y}_j) = \frac{(\mathbf{x}_i)^{T} \mathbf{y}_j}{ ||\mathbf{x}_i||_{2} \hspace{0.2em} ||\mathbf{y}_j||_{2}}
    = \cos(\theta_{i,j})
\end{equation}

is the normalised dot product between any two embeddings, where $\theta_{i,j}$ is the angle between the two embeddings.
The similarity measure $S_C$ is bounded in the range [-1, 1], and by construction, for any two embeddings which refer to
the same galaxy then $S_C(\mathbf{x}_i, \mathbf{y}_i)) = 1$ given that $\theta_{i,i} = 0 \hspace{0.25em} \forall i$.

\subsection{Data}\label{subsec:data}
We use the dataset as provided exactly by~\cite{astroclip}, with minor adjustments.
The dataset contains 197,976 galaxy spectra and image pairs, along with their redshift measurements.
The galaxy spectra were taken from the Dark Energy Spectroscopic Instrument (DESI) Early Data Release~\citep{desiearly2023}
and the galaxy images were taken from the DESI Legacy Survey~\citep{desilegacy2018}.
We summarise the key pre-processing steps relating to the data below.

\subsubsection{DESI Legacy Survey Images}\label{subsubsec:images}
The galaxy image dataset was curated by~\cite{stein2022} from the DESI Legacy Survey Data Release 9\footnote{https://www.legacysurvey.org/dr9/description/},
we refer the reader to~\cite{stein2022, astroclip} for a more comprehensive overview of the dataset and its curation, but the
key points are summarised here.
These images were taken by 3 different telescopes, with each telescope focusing on a different combination of sky area
and wavelength range, creating a survey of the sky with a sky coverage of 14,000 $deg^{2}$, at a pixel resolution of 0.262 arcseconds,
across the g (green: $4770 \si{\angstrom}$), r (red: $6231 \si{\angstrom}$), and z (infrared: $9134 \si{\angstrom}$) wavelength
bands\footnote{https://skyserver.sdss.org/dr1/en/proj/advanced/color/sdssfilters.asp}.
The Tractor\footnote{https://github.com/dstndstn/tractor} is used to probabilistically identify and infer properties such
as morphological classification of sources within the survey.
This creates a Tractor catalogue of each identified source, and a sweep catalogue is a subset of this information.
~\cite{stein2022} then filter the dataset as follows: they drop any source that were identified as a star in the sweep catalogues
(Tractor identified a best-fit morphological model of point-spread function, which indicates star); and drop any sources
with a z-band magnitude ($mag_{z}$) larger than 20.
Each remaining source is extracted through a 256x256 pixel cut-out centred on the source, in each of the (g, r, z) bands,
yielding a total of 76,446,849 images.
~\cite{astroclip} cross-match this dataset for their corresponding spectra from the DESI Early Data Release~\citep{desiearly2023}
using the targetIDs associated with the sources, to yield a final dataset of 197,976 galaxy spectra and image pairs.

We perform a variety of augmentations on the images including: a random horizontal and vertical shift; random rotation;
random horizontal and vertical flips as well as adding Gaussian noise.
The 256x256 pixel images all cover more sky than the source being analysed, so we follow~\cite{stein2022} in center
cropping the images to the central 96x96 pixels, adverse to the 144x144 cut-out performed by~\cite{astroclip}.
The choice of the 96x96 cut-out as opposed to a 144x144 cut-out is primarily due to the pre-training of our image embedder
as explained in Section~\eqref{subsubsec:image-embedder}.
The images are then standardised, ensuring that each channel (g, r, z) of each image had a mean
of 0 and a standard deviation of 1.
The Gaussian noise $\mathcal{G}(0, 0.03)$ was added on top of this standardised data such that each channel was noised
proportionately.


\subsubsection{DESI Early Data Release Spectra}\label{subsubsec:spectra}
The spectra were extracted from the DESI Early Data Release~\citep{desiearly2023} and only those that were successfully
cross-matched with the galaxy images were kept.
The spectra ranged the wavelength range of $[3600 \si{\angstrom}, 9824 \si{\angstrom}]$, with a resolution of 7781 pixels per spectrum.
As with the image data, the spectra were also standardised individually, to mean 0 and standard deviation 1, and then
noise $(\epsilon_{sp}(\lambda))$ was added in the form of scaled Gaussian noise as given in Equation~\eqref{eq:spectrum-noise}.

\begin{equation}
\label{eq:spectrum-noise}
    \epsilon_{sp}(\lambda) = \gamma \cdot \sigma_{sp}(\lambda) \cdot \mathcal{G}(0, 1)
\end{equation}
where $\epsilon_{sp}(\lambda)$ is the noise added to the spectrum at wavelength $\lambda$, $\gamma$ is a scaling factor
which is the same for all wavelengths, $\sigma_{sp}(\lambda)$ is the standard deviation of the spectrum at wavelength $\lambda$,
and $\mathcal{G}(0, 1)$ is a standard Gaussian random variable.
$\sigma_{sp}(\lambda)$ was computed for the training set such that $\sigma_{sp}(i)$ $(i \in [0, 7781])$ gave the standard
deviation of all spectra measurements at the $i^{th}$ pixel.
This meant that the noise added to the spectra at any given wavelength was proportional to the variance of the spectra
the given wavelength, which complemented the fact that some wavelengths naturally had more variance than others.
Finally, $\gamma$ was set to 0.3.

\subsubsection{Redshift measurements}\label{subsubsec:redshift}
The redshift measurements utilised were the catalog-reported measurements and were compiled and provided directly with the
dataset by~\cite{astroclip}.

\subsubsection{Further Pre-processing}\label{subsubsec:pre-processing}
The data was pre-processed further to remove outliers and ensure data was sensible to improve training dynamics.
We dropped any galaxies with a redshift outside the range $[0, 0.8]$ and also dropped any galaxies which had every element
in their spectra equal to 0 (given that they were clearly erroneous).
This resulted in removing a further 380 samples from the dataset, leaving a total of 197,596 galaxy spectra and image pairs.
Our redshift range of $[0, 0.8]$ was larger than the range of $[0, 0.6]$ used by~\cite{astroclip} in their final results.

\subsection{AstroCLIP Implementation}\label{subsec:astroclip-implementation}
Our AstroCLIP model was implemented using a pretrained 2D convolutional image embedder courtesy of~\cite{stein2021}
and a pretrained 1D spectrum embedder courtesy of~\cite{liang2023}.
These were placed into our unified AstroCLIP model and then fine-tuned (with all the weights set as trainable) using
contrastive learning under the InfoNCE loss as per Equation~\eqref{eq:infonce}, to learn the shared low-dimensional
embedding space.
Embeddings were projected into a variety of dimensions: (8, 16, 32, 64, 128, 256, 512) to explore the effect of the
dimensionality of the embedding space on the performance of the model, with~\cite{astroclip} using a 512 dimensional
embedding for their final results.
Below we give a brief overview of the pretraining strategies for the image and spectrum embedders, but for a
comprehensive report we refer the reader to the original papers.

\subsubsection{Image Embedder}\label{subsubsec:image-embedder}
\cite{stein2021} trained a galaxy image embedder using a self-supervised method for the purposes of image based similarity search.
The image embedder was trained via the MoCo v2 framework~\citep{moco2020, mocov22020} on a sample of 42 million galaxies
from the DESI Legacy Survey Data Release 9~\citep{desilegacy2018}.
The encoder architecture used was a ResNet-50 architecture.
The images underwent a variety of augmentations (galactic extinction, random rotation, blurring, noising) before
being cropped to the central 96x96 pixels.
The loss function used was the same InfoNCE loss we used in our AstroCLIP model as given in Equation~\eqref{eq:infonce},
and the similarity measure used was also the cosine similarity function as given in Equation~\eqref{eq:cosine-similarity}.
Their model yielded very promising results, and we refer the reader to the original paper for a thorough discussion
on the trained model.
The final fully-connected layers of this model contained 3 layers with 2048, 2048 and 128 units respectively.
The final 128 unit layer was adjusted as required to produce the output dimensionality we desired, but otherwise all
weights and biases were initiated as per the pretrained model.
This yielded between 27.7M and 28.8M trainable parameters depending on the output dimensionality of the model.

\cite{astroclip} use pre-train a vision transformer model for their image embedder, we refer the reader to the original paper
for a more detailed discussion on the architecture and pre-training strategy for this.

\subsubsection{Spectrum Embedder}\label{subsubsec:spectrum-embedder}
\cite{liang2023} trained a galaxy spectrum autoencoder using galaxy spectra from the DESI Early Data Release~\citep{desiearly2023},
focusing primarily on the Bright Galaxy Survey (BGS) (for precise details on the spectral dataset used, see the original paper).
The autoencoder architecture, named SPENDER, was proposed by~\cite{melchior2022} and encoded the spectra into a 6-dimensional
embedding space.
For our AstroCLIP model, we used the encoder part of the SPENDER model which consisted of a series of 1D convolutional layers
followed by a series of fully-connected layers to bring the dimensionality down to 6.
We reduced the fully-connected layers to 3 layers with 256, $D_{-1}$ and $D_{out}$ units respectively, for any desired
output dimensionality $D_{out}$.
The first 2 fully-connected layers in the pre-trained SPENDER encoder originally had a dimensionality of 256 and 128 units
and as such, for $D_{out} \in [8, 16, 32, 64, 128]$, we retained $D_{-1}=128$ and so the weights and biases of the
first 2 fully-connected layer were retained with the output layer being initialised from a Gaussian distribution.
For $D_{out} \in [256, 512]$, $D_{-1}$ was set to 256 and the entire fully-connected part of the model was retrained
from Gaussian initialisation.
The convolutional layers were initialised from the pre-trained weights of the SPENDER model.
This yielded between 3.1M to 3.3M trainable parameters, depending on the output dimensionality of the model.

\cite{astroclip} once again use a transformer architecture for their spectrum encoder, and we refer the reader to the original
paper for a more detailed discussion on the architecture and pre-training strategy for this.

\subsection{Training}\label{subsec:training}
Our AstroCLIP model unifies these two embedders into a single model, and then fine-tunes the embeddings using the InfoNCE loss
(Equation~\eqref{eq:infonce}).
Through some exploratory analysis, we fixed a few hyperparameters: batch size of 512; learning rate of $5e^{-4}$; a weight
decay of $1e^{-4}$; and a temperature parameter in the InfoNCE loss of 0.07.
The embedding dimensionality was varied through the range $[8, 16, 32, 64, 128, 256, 512]$ and each model was trained for
75 epochs.
After the training, the model with the lowest validation loss for each embedding dimensionality was selected for evaluation
in Section~\eqref{sec:results}.







%\subsection{Implementation}\label{subsec:implementation}
%Our implementation did not differ too greatly from the original AstroCLIP implementation~\citep{astroclip}.
%We created our AstroCLIP model with a pretrained image and spectrum embedder, and then fine-tune the model using contrastive
%learning under the InfoNCE loss to learn a shared 128-dimensional embedding space for the two modalities.
%To acquire the pretrained embedders, the fine-tuned embedders or the dataset used, we refer the reader to the referenced
%literature or to our codebase which contains all the necessary details.
%
%\paragraph{Spectrum Embedder} We acquired a pre-trained spectrum embedder from the works of~\cite{liang2023} who used
%a convolutional autoencoder to learn a 6-dimensional embedding for galaxy spectra using the Bright Galaxy Survey of the
%DESI Early Data Release~\citep{desiearly2023}, for precise details on architecture and training, see the original paper.
%The encoder consisted of a series of convolutional layers followed by 4 fully-connected layers.
%The last 3 of these layers were deleted and replaced with a single fully-connected layer to project the output of the first
%layer into a 128-dimensional embedding\footnote{That is to say, in the edited model, the first fully-connected layer took
%256 inputs and output 128-dims into the second layer which also output 128-dims.
%The first layer contained pre-trained weights, the second layer was randomly initialised.}.
%The entire model contained 3.2M parameters, all of which were set as trainable parameters, differing to the implementation
%by~\cite{astroclip} where the authors set only the final fully-connected layers of their pre-trained transformer-based spectral
%embedder as trainable.
%The spectra were normalised in the same way as done by~\cite{liang2023}\footnote{The spectra were normalised over their median
%flux in the rest-frame wavelength range of $[5300, 5850]$, where the rest-frame spectra were computed by assuming all
%spectra were redshifted by exactly $z=0.8$.} and then noise was added to the spectra using a
%similar technique as utilised by~\cite{astroclip}\footnote{Noise added to the spectra at any point was Gaussian and scaled
%by the distribution of values at any given flux measurement.
%Measurements at wavelengths with more variance therefore had more noise added.}.
%
%\paragraph{Image Embedder} The image embedder we used was the same as the one used by~\cite{astroclip}, courtesy of
%\cite{stein2021}, see Section~\eqref{sec:original-paper} for further details.
%However, unlike~\cite{astroclip} who only fine-tuned the fully-connected layers, we once again chose to set all 28.0M parameters
%of the model as trainable.
%The images were transformed and augmented using the same techniques used in the original AstroCLIP paper; the 3-channel (g, r, z)
%152x152 pixel images were augmented with Gaussian noise, rolled, rotated and flipped randomly before being cropped to the central
%96x96 pixels.
%This was then converted from (g, r, z) to RGB using the same scaling method utilised by~\cite{stein2021}.
%The conversion to RBG was relatively arbitrary and did not change meaning of the image, but was done to ensure the images
%were in the same format as expected by the pre-trained image embedder.
%
%\paragraph{Data} We use the same dataset as curated and helpfully provided by \cite{astroclip}, see Section~\eqref{sec:original-paper}
%for further details.
%This consists of 197,976 galaxy spectra and image pairs, along with their redshift measurements.
%The data was then further pre-processed to drop any galaxies with spectra outside the range $[0, 0.8]$ which was in line with the
%pre-training of the spectrum embedder, but is a larger range than utilised by \cite{astroclip} who used $[0, 0.6]$ in their final results.
%Additionally, some of the galaxy spectra in the dataset were completely flat (exactly 0 flux across all wavelengths)
%and so these galaxies were dropped from the dataset.
%This resulted in a total of 3,241 (1.6\%) image-spectra pairs being removed in the pre-processing step.
%The dataset was then split into a training and validation set, with 80\% of the data used for training and 20\% for
%validation; all results displayed in this paper were performed on the validation set.

%\paragraph{Training} ...
