%! Author = adnansiddiquei
%! Date = 04/06/2024

\section{Original AstroCLIP Paper}\label{sec:original-paper}
The original AstroCLIP paper~\cite{astroclip} whose results we reproduce in this paper, implemented a multi-modal contrastive
learning approach to embed galaxy spectra and galaxy images into a shared latent space.
This implementation utilised a transformer-based spectrum embedder and a convolutional image embedder to project the two
modalities into a shared 128-dimensional embedding, and further showed that the learned embeddings were well-aligned
with the physical properties of the galaxies.
The authors demonstrated that the learned embeddings could be used to make relatively accurate zero-shot predictions on
redshift and stellar mass of galaxies using k-NN regression.
For ease of comparison, we provide a brief overview of the implementation details and key results of the original
AstroCLIP paper.

\paragraph{Data} The galaxy image data used by the authors~\citep{astroclip} was curated from the DESI Legacy Survey Data
Release 9~\citep{desilegacy2018}, as prepared by~\cite{stein2021}.
This was cross-referenced to the DESI Early Data Release~\citep{desiearly2023} to obtain the corresponding galaxy spectra,
and cross-referenced to the PRObabilistic Value-Added Bright Galaxy Survey (PROVABGS) Catalog~\citep{provabgs2021}
to obtain the redshift and stellar mass labels for the galaxies - yielding a total of 197,976 galaxy spectra and image pairs.

\paragraph{Spectrum Embedder} The authors pre-trained a transformer-based spectrum embedder using a self-supervised mask
filling task on the galaxy spectra.
All the weights of this embedder were then frozen and a single cross-attention layer followed by an MLP was added to project
the spectrum into a 128-dimensional embedding, yielding a total of 362k trainable parameters.

\paragraph{Image Embedder} The pre-trained convolutional image embedder was acquired from~\cite{stein2021}.
This image embedder had a ResNet-50 architecture and was trained with a self-supervised method on a sample of 3.5 million
galaxy images (augmented in a variety of ways) from the DESI Legacy Survey~\citep{desilegacy2018} using the MoCo v2
framework~\citep{moco2020, mocov22020} in order to embed the images into a 128-dimensional space.
This image embedder had the convolutional layers frozen and yielding a total of 4.5M trainable parameters in the final
dense layers of the model.

\paragraph{Training} The authors used a contrastive learning approach to learn a shared embedding space for the images
and spectra.
A variety of augmentations were applied to the images and spectra and the respective embedders were trained using an
InfoNCE loss~\citep{infonce2020} over 15,000 iterations with a batch size of 512.

\paragraph{Key Results} The authors demonstrated that the embeddings were well-aligned in 2 main ways:
In-modal and cross-modal similarity searches using cosine similarity showed that two 'cosine similar' galaxies yielded
similar spectra and images.
Secondly, zero-shot regression of redshift and stellar mass using k-NN regression on the learned embeddings similarly
yielded relatively accurate predictions.
These are both further discussed in XXXXX when we compare the results of our reproduction with the original paper.
