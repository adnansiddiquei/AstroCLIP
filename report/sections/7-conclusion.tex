%! Author = adnansiddiquei
%! Date = 28/06/2024

\section{Conclusion}\label{sec:conclusion}
In this report, we present a successful reproduction of the AstroCLIP model by~\cite{astroclip}.
We utilise different architectures for the image and spectrum embedders, but ultimately achieve similar results
and show that the model is able to learn a meaningful representation of the data.
We also explore the effect of different embedding dimensionalities on the performance of the model and show that
even low-dimensional embeddings are able to capture a meaningful amount of mutual information between the two modalities,
given their strong performance in the zero-shot redshift prediction tasks.
Our results do bring into question the optimal architecture for the image embedder and the minimum required
embedding dimensionality, as our 128-dimensional model marginally outperforms the 512-dimensional model by~\cite{astroclip}
in the in-modal image to image redshift prediction task.
Nonetheless, our results alongside those of~\cite{astroclip} demonstrate that it is possible to achieve high quality foundation
models for astronomical data using cross-modal contrastive pre-training, and that the learned embeddings can be used for a variety
of downstream tasks with strong performance.
This has a variety of impacts on the field of astronomy, such as enabling cross-modal similarity searches for rare or interesting
objects, and enabling the use of pre-trained foundation models for transfer learning on smaller datasets for specific tasks.
