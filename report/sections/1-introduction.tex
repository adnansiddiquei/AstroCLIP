%! Author = adnansiddiquei
%! Date = 04/06/2024

\section{Introduction}\label{sec:introduction}
The size of scientific datasets, particularly in the field of astronomy, has been growing at an ever-increasing rate
over the last couple of decades.
Spectroscopic surveys such as the Sloan Digital Sky Survey (SDSS)~\citep{york2000} and more recently the Dark Energy
Spectroscopic Instrument (DESI)\footnote{https://www.desi.lbl.gov/} have been collating millions of galaxy spectra, while photometric
surveys such as the DESI Legacy Survey~\citep{desilegacy2018} has been imaging large portions of the sky extracting millions
of sources.
These datasets are used for a variety of scientific purposes, from understanding the large scale structure of the universe;
estimating galaxy properties such as redshift, stellar mass, and star formation rate; to identifying rare objects such as
quasars and supernovae; and many more.
However, the growing data set size and diversity makes much of this difficult and traditional methods are often
limited by the quality of the data and its associated labels.
One such example is morphological classification, a decade ago we had crowdsourced campaigns such as Galaxy Zoo 2~\citep{willet2013}
which classified approximately 300,000 galaxies, we now have tools such as Tractor\footnote{https://github.com/dstndstn/tractor}
which can probabilistically identify sources from photometric surveys and infer properties such as morphological classification.

More recently, given the unavailability of high quality labels, unsupervised and self-supervised learning methods have been
gaining popularity to tackle these sorts of tasks.
For example,~\cite{liang2023} train a 1D convolutional spectrum autoencoder on spectral data from the DESI Early Data
Release~\citep{desiearly2023} Bright Galaxy Survey for the purposes of outlier detection.
Similarly,~\cite{stein2021} train a 2D convolutional image embedder using a self-supervised technique on galaxy images
from the DESI Legacy Survey for the purposes of similarity search.
They use the Moco v2 self-supervised learning framework~\citep{moco2020, mocov22020}, which is a technique to learn image embeddings
by maximising the similarity of the embedding between augmented views of the same image, and minimising the similarity between
the embedding of different images.
~\cite{hayat2021} also use this technique to train a 2D convolutional model to estimate distances to galaxies from their
photometric images, and further demonstrate that the learned embeddings can be fine-tuned very effectively for redshift estimation.
They also show two important conclusions, (1) self-supervised pre-training followed by supervised fine-tuning can achieve
the same performance as supervised training from scratch, whilst requiring significantly fewer labels; and (2) fine-tuning
the self-supervised pre-trained model on the entire Main Galaxy Sample of the SDSS outperforms state-of-the-art supervised
learning methods.
These conclusions are not uncommon in the field of machine learning outside of astronomy
(NLP:~\cite{devlin2019, radford2018}; Medical Imaging:~\cite{shin2016}) and demonstrates the power of transfer learning
and the importance of foundation models for astronomical datasets.

However, as of yet most of these self-supervised learning methods have only been applied to a single modality despite
promising results in cross-modal contrastive learning outside of astronomy, such as contrastive language-image pre-training CLIP~\citep{radford2021}.
~\cite{astroclip} pioneer on this front by proposing a multi-modal contrastive learning approach to embed galaxy spectra
and galaxy images into a shared low-dimensional latent space, and in this paper we aim to reproduce their results.
Given the multi-modal nature of astronomical datasets, a useful astronomical foundation model should be able to embed
the varying views of the same object effectively into a shared latent space, allowing for in-modal and cross-modal downstream
application through zero-shot or few-shot learning.
